\documentclass[onecolumn]{hipatia}
\usepackage{blindtext}
\newcommand{\superau}{\textsuperscript{\underline{a}}~}
\title{Ampliação}
\subtitle{Epístola}
\author{}
\begin{document}
\setcounter{page}{\epistolapage}
\maketitle
\leftskip=2.5cm
\rightskip=2.5cm


\noindent Caro Leitor,
\vspace{1cm}


Informamos com alegria que nossa equipe técnica foi ampliada com a chegada de Álisson, Cleber, João Vítor, José Valdomiro e Yure. Sejam bem-vindos! Essa ampliação é fundamental para consolidar o projeto de extensão ao qual a Revista de Matemática Hipátia está vinculada. Nosso objetivo é que a revista cresça ainda mais e se torne um importante veículo de divulgação matemática no cenário local e, quem sabe, nacional.

Esta edição conta com um variado conjunto de temas. Em \textsc{História}, o prof. Marcelo Papini traça sua visão das origens do método axiomático. O convite para 
o prof. Papini escrever para nossa revista teve duas razões. A primeira é por
tratar-se de um especialista no assunto, tendo já publicado o livro ``Contribuição ao estudo histórico e crítico do pensamento matemático'', pela LF Editora.
A segunda razão é nossa admiração por sua prosa particular, 
percebida nas mensagens eletrônicas trocadas na lista de professores do Departamento
de Matemática da UFBA, do qual fazemos parte. 

Em \textsc{Técnica}, o prof. Carlos Augusto Ribeiro nos brinda com um 
artigo bastante completo --- e pioneiro em português --- sobre as Sequências de Farey, essa maneira 
curiosa de gerar todas as frações próprias. O prof. Carlos Augusto 
pode ser considerado um parceiro da Revista de Matemática Hipátia, tendo colaborado na edição anterior.

Na seção \textsc{Antologia}, apresentamos uma seleção de ditos 
sobre Matemática de vários matemáticos, filósofos e pensadores em geral, compilados por Alphonse Rebière, um matemático que 
se destacou pela defesa da participação feminina na ciência,
fato notável considerando a condição das mulheres em sua época. Em sua 
coletânea, Rebière faz uma 
justaposição interessante de citações, possibilitando um 
diálogo entre personagens de diferentes épocas.

Encerramos esta edição com as tradicionais seções \textsc{Simpósio},
a qual registra o vibrante cenário de eventos realizados no 
Departamento de Matemática da UFBA, e \textsc{Problema}, contendo
soluções de problemas da edição anterior e novos problemas.

\pagebreak

Na CAPA, temos uma superposição da lindíssima pintura ``O Casamento da Virgem'', também conhecida como \emph{Lo Sposalizio}, de Rafael Sanzio, expoente da renascença italiana, com uma representação
gráfica das Sequências de Farey, descrita em \textsc{Técnica}. 
Essa montagem, composta pelo prof. Nicola, poderia parecer arbitrária
 à primeira vista, mas os mestres do renascimento tinham
 tanto conhecimento intuitivo de Harmonia, que não espanta que,
 de algum modo, as Sequências de Farey estejam lá de fato.   

\vspace{1cm}

\hfill Salvador, 13 de dezembro de 2024

\hfill O Editor
\end{document}
